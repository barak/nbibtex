\documentclass[twocolumn]{article}
\title{A Selective History of Functional Programming\\
        {\large Or, How Good is \emph{Your} Bib{\TeX} File?}}
\author{Norman Ramsey}
\usepackage{natbib}
\usepackage{times}% smaller fonts
\bibpunct();A{},
\begin{document}
\maketitle
\small

This history is intended not so much to capture functional programming
as it is to test my replacement for Bib{\TeX}.
The idea is that you change the
\texttt{\textbackslash bibliography} command to use your own bibliography,
try my \texttt{nbibtex}, and see how many papers it picks up
correctly.
To get all~23 references, I~had to combine the bibliographies of four
professors (a 24th~reference was introduced by one of the~23). 
My own personal bibliography is missing 7~references
\citep{iverson::book,
  :lambda-ultimate-declarative,
  Backus:liberate-neumann,
  macqueen:MODULES-for-ml,
  milner-tofte-harper:definition:1990,
  jones:qualified-TYPES:booK,
  peyton-jones-launchbury:lazy-functional-state-threads:in}.




It all started with \cite{mccarthy:symbolic-recursive}.
\cite{iverson::book} had some good ideas, but what really got people
excited was Backus's \citeyearpar{Backus:liberate-neumann}
Turing lecture.
By that time of course we had Scheme \citep{sussman-steele:scheme:197},
which we all know is the ultimate language
 \citep{:lambda-ultimate-imperative,:lambda-ultimate-declarative,:lambda-ultimate-goto}.

But others were working eagerly on types.
\cite{milner:type-polymorphism-programming} was busy with~ML
and type inference \citep{damas-milner}.
MacQueen's \citeyearpar{macqueen:MODULES-for-ml} module proposal led to
Standard~ML \citep{milner-tofte-harper:definition:1990}.
Later work on modules by \cite{harper-lillibridge:sharing} led to
a revised Definition
\citep{milner-tofte-macqueen-harper}.
Closely related work by
\cite{leroy:modules:1994,leroy:functors:1995,leroy:modular-module-system}
led to new developments in the Objective Caml dialect.

Meanwhile, lazy functional programmers like
\cite{hughes:why-functional-matters} were busy developing Haskell 
\citep{peyton-jones:haskell:book}.
Nifty features like type classes
\citep{wadler-blott:ad-hoc-polymorphism} led to a cool theory of
qualified types \citep{jones:qualified-TYPES:booK}.
Not to mention monads \citep{wadler:essence-functional},
special threads
\citep{peyton-jones-launchbury:lazy-functional-state-threads:in},
and other awkward \citep{peyton-jones:awkward-squad} features.

There are too many papers to read them all in one semester, but when
my students held a vote among those they read, they liked
\cite{wadler-blott} best, although there was strong support for
\cite{leroy:separate-compilation}. 



\providecommand\includeftpref{} % wackiness of Simon's
\bibliographystyle{plainnatx}
\bibliography{fun}

\end{document}
